% -*-LaTeX-*-

\documentclass{sig-alternate-10pt}

\clubpenalty=10000 
\widowpenalty = 10000
\hyphenpenalty=5000
\tolerance=1000

%\usepackage{latexsym}
\usepackage[caption=false]{subfig}
\usepackage[sort]{cite}
\usepackage[hyphens]{url}
\usepackage{lastpage}
\usepackage{array, verbatim}
\usepackage{mdwlist}
\usepackage{multirow}
\usepackage{times}

\usepackage{hyperref,zref,xkeyval,pdfcomment}
\newcommand\todo[1]{\pdfcomment{#1}}
%\newcommand\todo[1]{}

\setlength{\pdfpagewidth}{8.5in}
\setlength{\pdfpageheight}{11in}

\usepackage{ifpdf}
\usepackage{graphicx}

\graphicspath{{./figs/}}

\ifpdf
  \DeclareGraphicsExtensions{.pdf}
\else
  \DeclareGraphicsExtensions{.eps}
\fi

%\newcommand{\enote}[2]{({\bf{#1:} \it{#2}})}
\newcommand{\enote}[2]{}

\def\full{0.8\linewidth}
\def\half{0.45\linewidth}
\def\third{0.30\linewidth}

\begin{document}



\title{Adding Heterogeneity to PlanetLab}

\numberofauthors{2}
\author{
\alignauthor
Angela H. Jiang\\
\affaddr{angelajiang@u.northwestern.edu}
\alignauthor
Zachary S. Bischof\\
\affaddr{Northwestern University}\\
\affaddr{zbischof@eecs.northwestern.edu}
}


\maketitle
%******************************************************************************
\begin{abstract}

PlanetLab has become a popular testbed for gathering Internet measurements. To
provide a more accurate platform for this purpose, researchers have attempted
to improve PlanetLab's representativeness of the Internet. While we don't know
what to model PlanetLab after, prior works have shown characteristics of these
nodes to be generally homogenous. We therefore focus on adding heterogeneity to
PlanetLab to increase its likeness to the Internet. We do this by exploiting
the existing heterogeneity of university networks in the world. Our work
attempts to discover how PlanetLab nodes are similar and university nodes with
the most potential to provide diversity.

\end{abstract}

%******************************************************************************
\section{Introduction}

PlanetLab is a network designed to connect large networks in academia. In this
way, a testbed is formed that is geographically distributed and collectively
managed. Although not it's original intention, many researchers take advantage
of this resource to conduct measures of the Internet.

However, due to the nature of it's design, qualities of PlanetLab sites are
found to be generally homogeneous. Sites are mostly connected to high speed
research networks located in the US and Europe. To allow PlanetLab to be a more
accurate Internet testbed, researchers have attempted to make the network more
representative of the Internet. A challenge to this is that we have an
incomplete view of what should be considered "representative". We therefore
focus on adding heterogeneity to PlanetLab to improve its likeness to the
Internet.

To function as a PlanetLab node, a site must meet certain resource and hardware
standards. For example, the site must be a server class machine and publicly
addressable by a static IP. These requirements allow PlanetLab to remain easily
managed and used but are mostly only met by University machines. There is
heterogeneity in university networks that is not represented in PlanetLab,
however. We attempt to pinpoint the universities that have the most potential
to increase PlanetLab diversity.  

Our goal is to first determine in what ways PlanetLab is homogenous. For this,
we look at number of hops, number of ASes and types of ASes traversed between
PlanetLab nodes. Next we compare potential PlanetLab sites organized by
geography, and determine in which location would nodes have a significant
impact on PlanetLab. 


%******************************************************************************
\section{Related Work} 

While measuring from well-connected hosts inside academic networks, Dischinger
et al.~\cite{dischinger:residential} characterized residential broadband
connections by sending packet trains to end-hosts. They included measurements
such as download and upload throughput, round-trip times and jitter, packet
loss rates, queue lengths, and queue drop policies. 

SatelliteLab~\cite{dischinger:satellitelab} 


%******************************************************************************
\section{Methodology}

\subsection{Selection of remote hosts}

%******************************************************************************
\section{Results}

Metrics used for comparison:

\begin{enumerate}
\item Latency
\item Throughput
\item Paths (AS paths?)
\item Number of ASes
\end{enumerate}


%******************************************************************************
\section{Conclusion}


%******************************************************************************
\bibliographystyle{abbrv}
\bibliography{diversity}

\end{document}
